\documentclass{article}
\usepackage{graphicx}
\usepackage{amsmath}
\usepackage[T2A]{fontenc}  % For Cyrillic fonts
\usepackage[utf8]{inputenc}  % For UTF-8 encoding
\usepackage[russian]{babel}  % For Russian language support
\usepackage{listings}
\usepackage{xcolor}
\usepackage{mathtools}
\usepackage{color}
\usepackage{hyperref}

\hypersetup{
    colorlinks=true,
    linkcolor=blue,
    filecolor=blue,      
    urlcolor=blue,
    citecolor=blue
}

\definecolor{dkgreen}{rgb}{0,0.6,0}
\definecolor{gray}{rgb}{0.5,0.5,0.5}
\definecolor{mauve}{rgb}{0.58,0,0.82}

\lstset{frame=tb,
    language=Python,
    aboveskip=3mm,
    belowskip=3mm,
    showstringspaces=false,
    columns=flexible,
    basicstyle={\small\ttfamily},
    numbers=left,
    numberstyle=\tiny\color{gray},
    keywordstyle=\color{blue},
    commentstyle=\color{dkgreen},
    stringstyle=\color{mauve},
    breaklines=true,
    breakatwhitespace=true,
    tabsize=3
}
\title{Report of Programming Task 1 of the course "Introduction to Optimization" - Fall 2024}
\author{Nikita Zagainov, Ilyas Galiev, Arthur Babkin, Nikita Menshikov, \\ Sergey Aitov}
\date{September 2024}

\begin{document}

\maketitle

\section{Team Information}
\noindent
Team leader: Nikita Zagainov - 5 \\
Team member 1: Ilyas Galiev - 5 \\
Team member 2: Arthur Babkin - 5 \\
Team member 3: Nikita Menshikov - 5 \\
Team member 4: Sergey Aitov - 5

\section{Link to the product}
\href{https://github.com/V1adych/simplex}{Project source code}

\section{Programming language}
Python

\section{Linear programming problem}
We aim to maximize nutritious value of salad given constraints on cost of its ingredients, maximum fats concentration, and weight of each individual component
\begin{center}
\begin{tabular}{|c||c|c|c|c|c|}
\hline

     Ingredient & Tomato & Cucumber & Bell Pepper & Lettuce Leaf & Onion \\
\hline \hline
     Cost, rub/kg & 130 & 100 & 155 & 85 & 50 \\
\hline
     Nutritious value, ckal/kg & 200 & 160 & 260 & 150 & 400 \\
\hline
     Max weight in salad, kg & 0.6 & 0.6 & 0.6 & 0.2 & 0.05 \\
\hline
     Fats, proportion & 0.004 & 0.005 & 0.006 & 0.003 & 0.004 \\ 
\hline
\end{tabular}
\end{center}
\begin{itemize}
    \item Maximization
    \item Objective function \& constraints:
\[
\text{maximize } c^T x
\]

subject to

\[
Ax \leq b
\]

where:

\[
A = \begin{bmatrix}
130 & 100 & 155 & 85 & 50 \\
0.004 & 0.005 & 0.006 & 0.003 & 0.004 \\
1 & 0 & 0 & 0 & 0 \\
0 & 1 & 0 & 0 & 0 \\
0 & 0 & 1 & 0 & 0 \\
0 & 0 & 0 & 1 & 0 \\
0 & 0 & 0 & 0 & 1
\end{bmatrix}
\]

\[
b = \begin{bmatrix}
200 \\
1 \\
0.6 \\
0.6 \\
0.6 \\
0.2 \\
0.05
\end{bmatrix}
\]

\[
c = \begin{bmatrix}
200 \\
160 \\
260 \\
150 \\
400
\end{bmatrix}
\]
\end{itemize}

\section{Output \& Results}
We tested our implementation of simplex method by comparing its outputs with \href{https://scipy.org}{scipy} implementation, and all tests show that outputs of both methods are the same on multiple tests, including original problem. \\
The method is applicable to our problem:
\begin{align*}
&\text{Problem is bounded: True} \\
&x: [0.2115, \ 0.6, \ 0.6, \ 0.2, \ 0.05] \\
&f: 344.3
\end{align*}

\section{Code}

\begin{lstlisting}[language=Python]
import numpy as np
from typing import Tuple


def simplex(
    A: np.ndarray, b: np.ndarray, c: np.ndarray, tol: float = 1e-6
) -> Tuple[bool, np.ndarray, float]:
    """
    Solves a linear programming problem using the simplex method.

    Args:
    A: A numpy array representing the coefficients of the constraints.
    b: A numpy array representing the right-hand side of the constraints.
    c: A numpy array representing the coefficients of the objective function.
    tol: A tolerance value for the simplex method.

    Returns:
    A tuple containing three elements:
    - A boolean value indicating whether the simplex method was successful.
    - A numpy array representing the solution to the linear programming problem.
    - The value of the objective function at the solution.
    """
    m, n = A.shape

    A_eq = np.hstack([A, np.eye(m)])
    c_eq = np.concatenate([c, np.zeros(m)])
    B = list(range(n, n + m))
    tableau = np.hstack([A_eq, b.reshape(-1, 1)])
    tableau = np.vstack([tableau, np.concatenate([c_eq, [0]])])

    while True:
        col = pivot_col(tableau, tol)
        if col == -1:
            break
        row = pivot_row(tableau, tol, col)
        if row == -1:
            return False, None, None

        tableau[row, :] /= tableau[row, col]
        for i in range(len(tableau)):
            if i != row:
                tableau[i, :] -= tableau[i, col] * tableau[row, :]

        B[row] = col

    x = np.zeros(n + m)
    x[B] = tableau[:-1, -1]

    return True, x[:n], c @ x[:n]


def pivot_col(tableau: np.ndarray, tol: float) -> int:
    last_row = tableau[-1, :-1]
    if np.all(last_row >= -tol):
        return -1
    return np.argmin(last_row)


def pivot_row(tableau: np.ndarray, tol: float, col: int) -> int:
    rhs = tableau[:-1, -1]
    lhs = tableau[:-1, col]
    ratios = np.full_like(rhs, np.inf)
    valid = lhs > tol
    ratios[valid] = rhs[valid] / lhs[valid]
    if np.all(ratios == np.inf):
        return -1
    return np.argmin(ratios)


def main():
    A = np.array(
        [
            [130, 100, 155, 85, 50],
            [0.004, 0.005, 0.006, 0.003, 0.004],
            [1, 0, 0, 0, 0],
            [0, 1, 0, 0, 0],
            [0, 0, 1, 0, 0],
            [0, 0, 0, 1, 0],
            [0, 0, 0, 0, 1],
        ],
        dtype=np.float32,
    )
    b = np.array([200, 0.01, 0.6, 0.6, 0.6, 0.2, 0.05], dtype=np.float32)
    c = -np.array([200, 160, 260, 150, 400], dtype=np.float32)
    state, x, f = simplex(A, b, c)
    print("Solver state:", "solved" if state else "not solved")
    print("Optimal solution:", x)
    print("Optimal value:", f)


if __name__ == "__main__":
    main()    
\end{lstlisting}
\end{document}
